\documentclass[reqno,a4paper,12pt]{amsart}

\usepackage{amsmath,amssymb,amsthm,geometry,xcolor,soul,graphicx}
\usepackage{titlesec}
\usepackage{enumerate}
%\usepackage{lipsum} used a paragraph to test the environment 
\usepackage{listings}
%\RequirePackage[most]{tcolorbox}
\usepackage{braket}
\allowdisplaybreaks[4] %align公式跨页
\usepackage{xeCJK}
\setCJKmainfont[AutoFakeBold = true]{Kai}
\geometry{left=0.7in, right=0.7in, top=1in, bottom=1in}

\renewcommand{\baselinestretch}{1.3}

\title{超导物理第五次作业}
\author{董建宇 ~~ 202328000807038}

%\setlength{\parindent}{2pt}

\begin{document}

\maketitle

\textbf{1.}从BCS基态能隙方程出发推导同位素效应。
\begin{proof}
考虑有限温度情况,能隙方程为
\[
	\Delta_k(T) = \sum_{k'} V_{kk'} \frac{\Delta_{k'}}{2E_{k'}} \left( 1 - \frac{2}{1+e^{\beta E_k}} \right).
\]

取近似$\Delta_k = \Delta_{k'} = \Delta, \ V_{kk'} = V$,则方程简化为:
\[
	1 = \frac{V}{2} \sum_k \frac{1}{\sqrt{\epsilon^2(k) + \Delta^2}} \tanh \frac{\sqrt{\epsilon^2(k) + \Delta^2}}{2k_BT}.
\]

求和化为积分,则有:
\[
	\frac{1}{N(0)V} = \int_0^{\hbar\omega_D} \frac{d\epsilon}{\sqrt{\epsilon^2 + \Delta^2}}\tanh \frac{\sqrt{\epsilon^2 + \Delta^2}}{2k_BT}.
\]

当$T = T_C$时,$\Delta(T_C) = 0$,则有:
\[
	\frac{1}{N(0)V} = \int_0^{\hbar\omega_D} \frac{d\epsilon}{\epsilon} \tanh\frac{\epsilon}{2k_BT_C} \approx \ln \frac{\hbar\omega_D}{2k_BT_C} + \ln \frac{4e^\gamma}{\pi}.
\]

整理可得:
\[
	k_BT_C = 1.13\hbar\omega_D e^{-\frac{1}{N(0)V}}.
\]

即$T_C \propto \omega_D \propto M^{-1/2}$,即超导转变温度与$\frac{1}{\sqrt{M}}$呈正相关。
\end{proof}

\medskip

\textbf{2.}从费米求的图像出发,说明两个电子总动量为零,形成库珀对时,由吸引位势造成能量降低最大。
\begin{proof}
在费米球上两总动量为零的电子$(k, -k)$的吸引位势使能量降低$V$。

若考虑两总动量不为零的电子$(k_1, k_2) = (k'+\delta k, -k'+\delta k)$,其中$k' = \frac{k_1+k_2}{2}, \ \delta k = \frac{k_1-k_2}{2}$,总可以选取一惯性系$S'$,使两电子在$S'$中总动量为零,$S'$即质心系,在$S'$系中,$(k'. -k')$电子吸引位势造成能量降低,但由于$S'$系相对实验室参考系有非零动量$\delta k$,则导致能量升高$\frac{1}{2}(2m)\left(\frac{\hbar\delta k}{m}\right)^2 = \frac{\hbar^2\delta k^2}{m} \geq 0$。

则为了最大程度降低能量,令$\delta k = 0$,即两电子总动量为零时,由吸引位势造成的能量降低最大。
\end{proof}

\medskip

\textbf{3.}试用测不准关系估计传统超导体Cooper Pair的尺寸。
\begin{proof}
记形成Cooper Pair的两个电子间距为$\xi$,则动量不确定度为$\delta p \sim \frac{\hbar}{\xi}$,动能不确定度为:
\[
	\delta\left( \frac{p^2}{2m^*} \right) = \frac{p}{m^*} \delta p \sim \frac{p_F}{m^*} \frac{\hbar}{\xi}.
\]

只有当$\delta\left( \frac{p^2}{2m^*} \right)<\Delta(0)$时,两电子才能形成稳定的Cooper Pair,即
\[
	\xi > \frac{\hbar p_F}{m^*\Delta(0)}.
\]
\end{proof}

\medskip

\textbf{4.}从有限温度下的BCS能隙方程,推导$T_C$附近能隙随温度变化的关系。
\begin{proof}
有限温度能隙方程为:
\[
	\Delta_k(T) = \sum_{k'} V_{kk'} \frac{\Delta_{k'}}{2E_{k'}} \left( 1- \frac{2}{1+e^{\beta E_k}} \right).
\]

取$\Delta_k(T) = \Delta_{k'}(T) = \Delta(T), \ V_{kk'} = V$,则有:
\[
	1 = \frac{V}{2} \sum_k \frac{1}{\sqrt{\epsilon^2(k) + \Delta^2}} - V \sum_k \frac{1}{\sqrt{\epsilon^2(k) + \Delta^2}} \times \frac{2}{1+\exp(\beta \sqrt{\epsilon^2(k) + \Delta^2})}.
\]

当$T\to 0$时,有:
\[
	1 = \frac{V}{2} \sum_k \frac{1}{\sqrt{\epsilon^2(k) + \Delta^2}} \simeq N(0)V \ln \frac{2\hbar\omega_D}{\Delta(0)}.
\]

即
\[
	N(0)V \ln \frac{2\hbar\omega_D}{\Delta(0)} = N(0)V \ln \frac{2\hbar\omega_D}{\Delta(T)} - 2N(0)V \int_0^{\hbar\omega_D} \frac{d\epsilon}{\sqrt{\epsilon^2(k) + \Delta^2}} \times \frac{1}{1+ \exp(\beta\sqrt{\epsilon^2(k) + \Delta^2})}.
\]
\[
	\Longrightarrow \ln \frac{\Delta(0)}{\Delta(T)} = 2\int_0^{\hbar\omega_D} \frac{d\epsilon}{\sqrt{\epsilon^2(k) + \Delta^2}} \times \frac{1}{1+ \exp(\beta\sqrt{\epsilon^2(k) + \Delta^2})}.
\]

当$\vert T_C - T \vert \ll T_C$时,
\[
	\Delta(T) \propto \Delta(0) \sqrt{\frac{8}{7\xi(3)}}\left( 1- \frac{T}{T_C} \right)^{1/2} \propto \left( 1- \frac{T}{T_C} \right)^{1/2}.
\]
\end{proof}


\end{document}